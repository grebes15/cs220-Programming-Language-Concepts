\documentclass{article}

\usepackage{tabto}
\usepackage{enumitem}
\begin{document}
\noindent
Andreas Landgrebe
\\
Computer Science 220
\\
Lab 4: Controlling Structures
\\
Part 2: Write a step-by-step description of the bytecode
\\
\begin{enumerate}
\NumTabs{6}
\item \textbf{bipush 10} \tab{push the constant 10 onto the stack}
\\
\item \textbf{istore\_1} \tab{pop the 10 and save it in location 1 (i = 10)}
\\
\item \textbf{bipush 20} \tab{push 20 onto the stack}
\\
\item \textbf{istore\_2} \tab{[pop the 20 and save it in location 2 (j = 20)}
\\
\item \textbf{iconst\_0} \tab{push 0 onto the stack}
\\
\item \textbf{istore\_3} \tab{pop 0 and save it in location 3 (k = 0)}
\\
\item \textbf{iload\_1} \tab{push contents of location 1 onto stack (i)}
\\
\item \textbf{bipush 10} \tab{push constant 10 onto stack}
\\
\item \textbf{if\_icmple 15} pop two element and compare with ``\textless='' (if(i \textless= 10))
\\
\tab{if test is true, go to line 15}
\\
\item \textbf{iload\_2} \tab{[test was false] load contents of location 2 onto stack (j)}
\\
\item \textbf{bipush 20} \tab{push the constant 20 onto stack}
\\
\item \textbf{if\_icmpne 15} pop two elements and compare with ``!='' (if j !== 20)
\\
\tab{if test is true, go to line 15}
\\
\item \textbf{bipush 100} \tab{[test was false] push the constant 100 onto the stack}
\\
\item \textbf{istore\_3} \tab{pop stack and store in location 3(k = 100)}
\\
\item \textbf{iload\_3} \tab{push contents of location 3 onto stack (return k)}
\\
\item \textbf{ireturn}  \tab{}

\end{enumerate}
\end{document}